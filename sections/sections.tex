\documentclass[a4paper,12pt]{article}
\usepackage{parskip}

\title{How to add a title, sections and a contents page}
\date{\today}
\author{Anas Syed}

\begin{document}
\maketitle
\newpage % The \newpage command creates a new page. Here, it is used to ensure the title has its own page. It is optional here.
\tableofcontents
\newpage % This ensures that the table of contents also has its own page. Try removing this and see what happens.

\section{This is the first section}
To make a title, you need to do two things. First, add at least one of the following lines to the preamble:

\begin{verbatim}
\title{How to add a title, sections and a contents page}
\date{\today}
\author{Anas Syed}
\end{verbatim}

Then, add the \verb|\maketitle| command anywhere in the document. You could even have the title at the bottom!

To add the contents page, just add the \verb|\tableofcontents| command anywhere in the document.

\subsection{This is a subsection heading}
This is the subsection content. You can use sections, subsections, subsubsections and paragraphs to organise your document.

\subsubsection{This is a subsubsection heading}
This is the subsubsection.

\subsection{This is another subsection}
Hello

\paragraph{This is the paragraph title}
This is the paragraph content.

\section{This is another section}
Notice how the sections are numbered:
\subsection{Subsection}
This subsection is numbered 2.1. If you don't want to number a section, just put an asterisk after the command, like so:

\begin{verbatim}
\subsection*{Another subsection}
\end{verbatim}

\subsection*{Another subsection}
Here is another subsection, but here you can see it has not been numbered due to the use of an asterisk.

\subsection{A third subsection}
As you can see, this subsection is numbered as 2.2, because the previous subsection skipped the numbering.

\end{document}
