\documentclass[a4paper,12pt]{article}
\usepackage{parskip}
\usepackage{amsmath} %Always include this when using maths - it has some very important features.
\usepackage{hyperref} %For hyperlinks to websites. This package automatically makes all references into clickable hyperlinks

\begin{document}
Maths is the most important feature in \LaTeX. Inline maths is created using dollar signs, e.g.\ $e^{i\theta}=\cos\theta + i\sin\theta$. It is highly advisable to include the \texttt{amsmath} package when using maths.

Equations on their own line are created using the \texttt{equation} environment:

\begin{equation}
  \sigma_{xx} =
  \begin{bmatrix}
    \sigma_{xx} & \tau_{xy}   & \tau_{xz}   \\
    \tau_{yx}   & \sigma_{yy} & \tau_{yz}   \\
    \tau_{zx}   & \tau_{zy}   & \sigma_{zz} \\
  \end{bmatrix}
  \label{eqn:stressTensor}
\end{equation}

Equation \ref{eqn:stressTensor} shows an example of a stress tensor equation.

Maths in \LaTeX\ is a long and complicated topic, so refer here \url{https://en.wikibooks.org/wiki/LaTeX/Mathematics}
\end{document}
